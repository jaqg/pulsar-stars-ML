\appendix
% ------
\section*{Appendix}
\label{sec:appendix}
% ------
% ------
\section{Pulsars}
% ------
Pulsars are a rare type of Neutron star that, as they (rapidly) rotate, produce 
a detectable pattern of broadband radio emission, which repeats periodically.
Each pulsar produces a slightly different emission pattern, which varies slightly
with each rotation. 
Thus, a potential signal detection, known as a ``candidate'', is averaged over many
rotations of the pulsar.
In the absence of additional info, each candidate could potentially describe a
real pulsar.
However, in practice almost all detections are caused by radio frequency
interference (RFI) and noise, making legitimate signals hard to find
~\cite{2016MNRAS.459.1104L, 2010MNRAS.409..619K, 2004hpa..book.....L}.
Here is when machine learning algorithms come in hand to automatically label
pulsar candidates, which is the purpose of this report.

