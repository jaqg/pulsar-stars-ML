\graphicspath{{./../script/pulsar-stars/}}
% short document (1-2 pages) describing what you did and the data you used.

% ------
\section{Introduction}
% ------
Support Vector Machine (SVM), or support-vector network, is a supervised machine
learning algorithm used for two-group classification and regression problems,
which conceptually implement the following idea: input vectors are non
linearly mapped to a very high-dimension feature space, in which a linear
decision surface is constructed~\cite{cortes1995supportvector}.
% It is constructed by combining: the solution technique from optimal
% hyperplanes (that allows for an expansion of the solution vector on support
% vectors), the idea of convolution of the dot-product (that extends the
% solution surfaces from linear to non-linear), and the notion of soft margins 
% to allow for errors on the training set)~\cite{cortes1995supportvector}.

It was originally implemented for the restricted case where the training data
can be separated without errors, but it has been extended to non-separable
training data, as is the case of the data used in this report.
After testing linear and non-linear 
kernels~\cite{boser1992training, aizerman2019theoretical}, the Gaussian Radial
Basis Function (RBF)~\cite{fornberg2011stable} kernel was used to construct the
SVM.

% ------
\section{Data}
% ------
The chosen dataset is HTRU2~\cite{misc_htru2_372},
which can be found in the
\href{https://archive.ics.uci.edu/dataset/372/htru2}{UCI Machine Learning Repository}
and in
\href{https://www.kaggle.com/datasets/charitarth/pulsar-dataset-htru2}{kaggle}.
HTRU2 is a data set which describes a sample of pulsar candidates collected
during the High Time Resolution Universe Survey (South)~\cite{2016MNRAS.459.1104L}.

Pulsars are a rare type of Neutron star that, as they (rapidly) rotate, produce 
a detectable pattern of broadband radio emission, which repeats periodically.
Each pulsar produces a slightly different emission pattern, which varies slightly
with each rotation. 
Thus, a potential signal detection, known as a ``candidate'', is averaged over many
rotations of the pulsar.
In the absence of additional info, each candidate could potentially describe a
real pulsar.
However, in practice almost all detections are caused by radio frequency
interference (RFI) and noise, making legitimate signals hard to find
~\cite{2016MNRAS.459.1104L, 2010MNRAS.409..619K, 2004hpa..book.....L}.
Here is when machine learning algorithms come in hand to automatically label
pulsar candidates, which is the purpose of this report.

The dataset contains a total of 17898 samples, where 16259 are spurious examples
caused by RFI/noise, and 1639 are real pulsar examples.
Each candidate is described by 8 continuous variables, which correspond to
simple statistic obtained from the integrated pulse profile and from the 
dispersion signal-to-noise ratio (DM-SNR) curve.
In the last entry, the class labels are 0 (negative) and 1 (positive):

\begin{multicols}{2}
    \begin{enumerate}
        \item Mean of the integrated profile.
        \item Standard deviation of the integrated profile.
        \item Excess kurtosis of the integrated profile.
        \item Skewness of the integrated profile.
        \item Mean of the DM-SNR curve.
        \item Standard deviation of the DM-SNR curve.
        \item Excess kurtosis of the DM-SNR curve.
        \item Skewness of the DM-SNR curve.
    \end{enumerate}
\end{multicols}

Summary of the HTRU2 dataset samples:

\begin{multicols}{3}
    \begin{itemize}
        \item 17898 total.
        \item 1639 (9.16\%) positive.
        \item 16259 (90.84\%) negative.
    \end{itemize}
\end{multicols}

In order to visualize the data, the pairplot of the dataset is shown in
\ref{fig:pairplot}.

\begin{figure}[b!]
    \centering
    \includegraphics[width=1.0\textwidth]{pairplot.jpg}
    % \subimport{figures/}{../script/giants-dwarfs/pairplot.pdf.pgf}
    \caption{Pairplot of the HTRU2 dataset.}
    \label{fig:pairplot}
\end{figure}

Looking at the positive $\left( \approx 10\% \right)$ \textit{vs} negative 
$\left( \approx 90\% \right)$ ratio and the pairplot in \cref{fig:pairplot},
the raw data is imbalanced.
It can be confirmed simply by eye from the Principal Component Analysis (PCA)
plotted in \cref{fig:pca}.

\begin{figure}[b!]
    \centering
    \includegraphics[width=1.0\textwidth]{pca.pdf}
    % \subimport{figures/}{../script/giants-dwarfs/pairplot.pdf.pgf}
    \caption{PCA plot of the HTRU2 dataset.}
    \label{fig:pca}
\end{figure}
% ------
\section{Program}
% ------

% ------
\section{Conclusions}
% ------
% Advantages of using SVM in Python include its ability to handle complex and high-dimensional data, achieve high accuracy, and be less sensitive to outliers.
% Other characteristics like capacity control and ease of changing the implemented decision
% surface render the support-vector network an extremely powerful and universal learning
% machine~\cite{cortes1995supportvector}.
